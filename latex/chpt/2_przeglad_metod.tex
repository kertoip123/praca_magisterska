\chapter{Przegląd metod detekcji obiektów pierwszoplanowych}
\label{cha:przeglad_metod}

\section{Wprowadzenie}
\label{sec:przeglad_wprowadzenie}

Niniejszy rozdział ma na celu przedstawić dotychczasowe osiągnięcia w dziedzinie segmentacji obiektów pierwszoplanowych. Ponieważ jest to zagadnienie bardzo rozległe, położono nacisk głównie na najpopularniejsze podejścia, osiągające zadowalające rezultaty w różnego rodzaju testach weryfikacyjnych. Omówiono zarówno algorytmy opracowane w~ramach prac dyplomowych oraz artykułów w~Laboratorium Biocybernetyki AGH jak i~inne metody opublikowane podczas konferencji naukowych na całym świecie. 

Główny nacisk położono na pokazanie różnych podejść do tworzenia zaawansowanych modeli tła, wykorzystywanego następnie do klasyfikacji pikseli znajdujących się na kolejnych ramkach obrazu. Niezależnie od samego sposobu modelowania tła, istotnym zagadnieniem jest także sposób jego aktualizacji. Różne podejścia do tematu aktualizacji używanego modelu również zostały przedstawione w niniejszym rozdziale. Kolejnym elementem wartym zaznaczenia, są różnego rodzaju dodatkowe funkcjonalności wspierające podstawową wersję algorytmu. Może to być, na przykład moduł do wykrywania obiektów statycznych, bądź też mechanizm eliminujący drgania kamery.


\section{Model tła bazujący na poprzednich ramkach}
\label{sec:model_poprzednie_ramki}

Pierwszym typem algorytmów, są metody bazujące na poprzednich ramkach obrazu. Jednym z najprostszych tego typu algorytmów jest tzw. ,,średnia krocząca''. Metoda ta została zrealizowana w niniejszej pracy, opis teoretyczny zamieszczono w rozdziale \ref{sec:proste_metody}. Do tego typu algorytmów możemy zaliczyć także prostą metodę odejmowania ramek oraz porównywanie aktualnej ramki z modelem referencyjnym. Obie te metody zaprezentowano w przytoczonym wyżej rozdziale.

Spośród zaawansowanych algorytmów, których model opiera się na poprzednich ramkach, niewątpliwie jedną z bardziej popularnych metody jest \textit{ViBE} (\textit{Visual Background Extractor}). Algorytm został opisany w wielu publikacjach, między innymi w \cite{barnich_11, droogenbroeck_12}, doczekał się także wielu usprawnień. Oprócz zagranicznych publikacji metoda ta, była również obiektem badań w Laboratorium Biocybernetyki AGH, gdzie dokonano jej implementacji w układzie reprogramowalnym \cite{kryjak_13_vibe}. Oprócz standardowej wersji opracowano także rozszerzoną odmianę algorytmu, posiadającą dodatkowy mechanizm eliminacji efektów drgającej kamery \cite{kryjak_14_vibe}. Metoda ta została również zawarta w niniejszej pracy, jej szczegółowy opis przedstawiono w rozdziale \ref{sec:vibe_teoria}.

Kolejnym algorytmem bazującym na poprzednich ramkach obrazu jest \textit{PBAS} (\textit{Pixel Based Adaptive Segmenter}). Metoda została zaprezentowana między innymi w \cite{hofmann_12}. W ramach badań w Laboratorium Biocybernetyki AGH opracowano implementację sprzętową tego algorytmu \cite{kryjak_13_pbas}. Dodatkowo udało się dodać do podstawowej wersji algorytmu dodatkowych mechanizm zapewniający lepszą detekcję obiektów statycznych \cite{kryjak_14_pbas}. Ulepszona wersja również została zaimplementowana w układzie \textit{FPGA}. Omawiany algorytm również jest elementem badań niniejszej pracy, a jego dokładny opis teoretyczny zamieszczono w rozdziale \ref{sec:pbas_teoria}.

\section{Inne rodzaje modeli tła}
\label{sec:model_inne}

Drugą grupą algorytmów, są metody bazujące w większym stopniu na modelach statystycznych. Pierwszą, najprawdopodobniej najpopularniejszą metodą tego rodzaju jest algorytm \textit{GMM} (\textit{Gaussian Mixture Models}). Głównym założeniem tej metody jest reprezentacja modelu tła poprzez rozkłady Gaussa. Algorytm po raz pierwszy został opisany w 1999 roku w publikacji \cite{Stauffer_Grimson_99}. Powstały również publikacje przedstawiające implementację tej metody w układzie reprogramowalnym \cite{Genovese_Napoli_13}. Algorytm, był także przedmiotem badań prac dyplomowych w Laboratorium Biocybernetyki AGH \cite{piszczek_15, janus_15}. Przygotowana w ramach pracy dyplomowej \cite{piszczek_15} implementacja sprzętowa została również przeanalizowana w niniejszej pracy. Szczegółowy opis algorytmu zamieszczono w rozdziale \ref{sec:gmm_teoria}.

Kolejnym algorytmem przedstawiającym nieco odmienne podejście jest metoda \textit{Flux Tensor}. Jest to algorytm bazujący na wykrywaniu krawędzi obiektu i badaniu pochodnej obrazu po czasie w celu wykrycia obiektów ruchomych, niestety przy jej użyciu nie ma możliwości detekcji obiektów statycznych. Od strony teoretycznej, metoda została przestawiona między innymi w publikacji \cite{palaniappan_11}, natomiast pierwsza implementacja sprzętowa została opracowana w Laboratorium Biocybernetyki AGH i przedstawiona na konferencji \textit{ICCVG} (\textit{International Conference on Computer Vision and Graphics}) \cite{janus_15, janus_16_flux}. 

Bardzo ciekawe podejście do zagadnienia segmentacji obiektów zostało przedstawione w publikacji \cite{wang_14}. Pokazany algorytm \textit{FTSG} (\textit{Flux Tensor with Split Gaussian models}) jest metodą hybrydową wykorzystującą omówione wcześniej metody \textit{GMM} i \textit{Flux Tensor} oraz dodatkowe mechanizm detekcji dynamicznego tła i obiektów statycznych. Algorytm w momencie publikacji był najdokładniejszą metodą w rankingu \textit{ChangeDetection} \cite{change_detection_web}. Metoda ta, została częściowo zaimplementowana w układzie reprogramowalnym w ramach jednej z pracy dyplomowych w Laboratorium Biocybernetyki AGH \cite{janus_15}.

Do grupy metod statystycznych można zaliczć także algorytm \textit{KDE} (\textit{nonparametric Kernel Density Estimation}). Metoda została po raz pierwszy opublikowana w 2002 roku \cite{elgammal_02}. Rezultaty, które można uzyskać w testach przy jej wykorzystaniu nadal są satysfakcjonujące przez co często jest cytowana w~różnego rodzaju publikacjach i służy jako punkt odniesienia do nowych algorytmów. Idea metody opiera się na funkcji gęstości prawdopobieństwa, która jest używana do stworzenia modeli statystycznych tła i~pierwszego planu. Podobnie jak w przypadku pozostałych algorytmów, ta metoda również doczekała się wielu rozszerzeń i usprawnień \cite{nonaka_12}.

\section{Różne podejścia do aktualizacji modelu tła}
\label{sec:model_tla_aktualizacja}

W przypadku większości algorytmów wykorzystujących zaawansowany model tła, pojawia się zagadnienie aktualizacji modelu. Temat ten, został poruszony w większości publikacji przytoczonych w~rozdziałach \ref{sec:model_poprzednie_ramki} i \ref{sec:model_inne}. Autorzy zazwyczaj rozróżniają dwa rodzaje podejść do aktualizacji modelu: liberalne (ang. \textit{liberal approach}) oraz podejście konserwatywne (ang. \textit{conservative approach}). Podejście liberalne zakłada aktualizację wszystkich modeli, podczas gdy konserwatywne jedynie tych pikseli, które zostały sklasyfikowane jako tło. Obie metody mają oczywiście swoje wady i zalety.

Główną wadą podejścia liberalnego jest stosunkowo szybkie wtapianie się obiektów pierwszoplanowych do modelu tła. Zjawisko to, jest szczególnie widoczne w przypadku wolno poruszających się obiektów. Polityka konserwatywnego aktualizowania modelu eliminuje ten problem, jednak ma inną poważną wadę. Podejście to, może prowadzić do omówionego już, wystąpienia tzw. ,,duchów'', czyli błędnej interpretacji odsłoniętego tła. Tego typu przypadek, może wystąpić w przypadku gdy obiekt pierwszoplanowy, początkowo statyczny (np. samochód stojący na światłach), zaczyna się porusząć. W~takiej sytuacji algorytm może nadal interpretować pozostawiony po samochodzie obszar jako element pierwszego planu. Kolejnym problemem są również różnego rodzaju zakłócenia i szumy, raz błędnie sklasyfikowany obszar może pozostać już nienaprawiony.

Po analizie wspomnianych publikacji można zauważyć, że pomimo poważnych wad, częściej stosowanym podejściem jest polityka konserwatywna. Wykorzystywane są również dodatkowe mechanizm pomagające eliminować wspomniane problemy. Przykładem może być niezależna aktualizacja pikseli sąsiadujących z aktualnie aktualizowanym jak ma to miejsce w algorytmach \textit{ViBE} \cite{kryjak_13_vibe} lub \textit{PBAS} \cite{kryjak_13_pbas}. Innym rozwiązaniem jest osobny moduł do całkowitej eliminacji zjawiska ,,duchów'', takie podejście zrealizowano między innymi w rozszerzonym algorytmie \textit{PBAS} \cite{kryjak_14_pbas} oraz \textit{FTSG} \cite{wang_14}.

\section{Podsumowanie}
\label{sec:przeglad_podsumowanie}

Powyższy rozdział, został napisany w celu zaprezentowania różnych podejść do tematu segmentacji obiektów pierwszoplanowych. Jak łatwo zauważyć, na przykładzie wymienionych metod, jest to dziedzina bardzo rozległa oraz stale się rozwijająca. Pokazane algorytmy dowodzą, że do tego zagadnienia można podejść na wiele sposobów. Prezentacja nowych algorytmów lub usprawnionych wersji metod już dostępnych zazwyczaj ma miejsce na rożnego rodzaju corocznych konferencjach. Do najpopularniejszych możemy zaliczyć \textit{AVSS} (\textit{Advanced Video an Signal -- Based Surveillance}, \textit{CVPR} (\textit{Computer Vision and Patter Recognition}) oraz \textit{ICCVG} (\textit{International Conference on Computer Vision and Graphics}).

Ponieważ dokładny opis wszystkich algorytmów znajduję się we wskazanych artykułach, w tym rozdziale przedstawiono jedynie idee, oraz główne założenia poszczególnych metod. Warto zaznaczyć, że w niektórych przypadkach istnieje wiele podejść do realizacji tej samej metody. Takim przykładem, jest między innymi metoda \textit{GMM}. Drugim elementem wartym uwagi, jest sam dobór i kalibracja konkretnej metody do pracy w konkretnym środowisku, dyskusja na tym polu również może być bardzo rozległa. 