\chapter{Dalsze kierunki rozwoju}
\label{cha:kierunki_rozwoju}

\section{Poprawa algorytmów}
\label{sec:poprawa_algorytmow}

Oprócz standardowych wersji algorytmów, udało się także zaimplementować ich rozszerzone wersję, zdecydowanie poprawiając efektywność w specyficznych warunkach. W przypadku algorytmu \textit{ViBE} zaimplementowano moduł, który w znacznym stopniu eliminuje efekt drgającej kamery. Metoda \textit{PBAS} została natomiast rozszerzona o funkcjonalność detekcji obiektów statycznych. Kolejnym krokiem w~rozwoju algorytmów i zwiększania ich efektywności mogłoby być zaimplementowanie metody zawierającej oba te udoskonalenia. Dobrym punktem wyjścia może być w tym przypadku algorytm \textit{PBAS}, zawierający już moduł detekcji obiektów statycznych. Dodanie do takiej implementacji modułu wyliczającego przepływ optyczny powinno zostać wykonane bez dużych komplikacji. 

W Laboratorium Biocybernetyki AGH, w ramach pracy inżynierskiej \cite{janus_15} został częściowo zaimplementowany algorytm \textit{FTSG}. Jest to pierwsza implementacja tej metody w układzie reprogramowalnym. Nie udało się niestety opracować pełnej wersji, zaproponowanej przez autorów oryginalnej publikacji \cite{wang_14}. Zabrakło między innymi mechanizmu detekcji obiektów statycznych, który według założeń miał być bardzo podobny do tego wykorzystanego w metodzie \textit{PBAS}. Celem przyszłych badań może być próba wykorzystania opracowanego dla algorytmu \textit{PBAS} moduł analizy obiektów \textit{CCA} i zintegrowania go z algorytmem \textit{FTSG}. W celu poprawny dokładności można spróbować także zaimplementować mechanizm indeksacji obiektów w wariancie dwuprzebiegowym.

\section{Wzrost wydajności}
\label{sec:wzrost_wydajnosci}

Algorytmy przedstawione w niniejszej pracy starano się uruchomić w zarówno niskich (\textit{720x576}) jak i wysokich (\textit{1280x720}, \textit{1920x1080}) rozdzielczościach. Zamierzony cel udało się osiągnąć w przypadku podstawowych wersji algorytmu, niezawierających dodatkowych modułów odpowiedzialnych między innymi za redukcję drgań kamery lub detekcję obszarów statycznych. W wielu przypadkach wiązało się to niestety z obniżeniem dokładności algorytmu, na przykład z powodu redukcji modelu tła, bądź konieczności przejścia z przestrzeni kolorów \textit{RGB} do skali szarości. 

W przypadku rozszerzonych odmian algorytmów \textit{ViBE} oraz \textit{PBAS}, udało się jedynie zapewnić przetwarzanie w najniższej rozdzielczości. Istotną kwestią jest zatem optymalizacja przygotowanych implementacji sprzętowych. Wraz z postępem technologicznym i pojawianiem się nowych, wydajniejszych układów \textit{FPGA}, należy dążyć do obsługi wyższych rozdzielczości. Docelowo system wizyjny powinien przetwarzać obraz w rozdzielczości \textit{1920x1080} w 50 klatkach na sekundę, w dalszej przyszłości wymaganym standardem może stać się rozdzielczość \textit{4K (3840x2160 pikseli)}.

Wzrost wydajności systemów wizyjnych jest ograniczony przez kilka czynników. Jak zostało już podkreślone jedną z blokad jest aktualnie dostępny sprzęt. W niektórych przypadkach ujawnia się ograniczona ilość zasobów logicznych w układzie \textit{FPGA} lub dostępna pamięci \textit{RAM}. Jednak, aby zapewnić wzrost wydajności nie należy jedynie oczekiwać na postęp technologiczny. Każdy algorytm można w~pewnym stopniu zoptymalizować, poprzez uproszczenie logiki i zredukowanie ilości operacji wykonywanych w jednym cyklu. Taki zabieg może co prawda zwiększyć sumaryczną latencje, ale jednocześnie zapewni wyższą maksymalną częstotliwość pracy zegara. 


\section{Implementacja nowych rozwiązań}
\label{sec:implementacja_nowych_rozwiazan}

Mimo stosunkowo zadowalających efektów końcowych przygotowanych implementacji, nadal istnieje wiele kierunków w których algorytmy powinny być rozwijane. W przyszłości należy zastanowić się nad przygotowaniem między innymi dodatkowego mechanizmu detekcji dynamicznego tła i eliminacji jego wpływu na pracę algorytmu. Kolejnym wartym uwagi zagadnieniem jest prawidłowa detekcja cieni i ich odróżnienie od rzeczywistych obiektów. Oba zagadnienia są tematami bardzo rozległymi, które nie zostały do tej pory poruszone w ramach badań w Laboratorium Biocybernetyki AGH.

Kolejną drogą rozwoju może być próba tworzenia implementacji sprzętowych istniejących już algorytmów. Autorzy publikacji na różnego rodzaju konferencjach poświęconych systemom wizyjnym prezentują nowatorskie rozwiązania, często jednak nowy algorytm przedstawiany jest jedynie od strony teoretycznej. Mimo, że model programowy przygotowany na komputerze klasy PC daje świetne rezultaty w testach, jego zastosowanie w rzeczywistym systemie wizyjnym jest niemożliwe dopóki nie powstanie dedykowana implementacja sprzętowa. Przykładem takiego działania może być algorytm \textit{Flux Tensor}, którego pierwsza implementacja sprzętowa \cite{janus_16_flux} została przygotowana właśnie w Laboratorium Biocybernetyki AGH.