\chapter{Zakończenie}
\label{cha:zakonczenie}

Wszystkie postawione w rozdziale \ref{sec:wprowadzenie_cel_pracy} udało się zrealizować, zatem można stwierdzić, że badania wykonane w ramach niniejszej pracy zakończyły się sukcesem. Pierwszym krokiem było przygotowanie listy algorytmów, które miały zostać zaimplementowane na docelowej platformie. Zgodnie z założeniami pracy zdecydowano się wybrać metody, będące już częściowo opracowane w ramach badań prowadzonych w Laboratorium Biocybernetyki AGH. Wybrany zestaw zawiera algorytmy prezentujące różnorodne podejścia do zagadnienia detekcji obiektów pierwszoplanowych i segmentacji tła. W stworzonej bibliotece zawarto zarówno bardzo proste obliczeniowo metody jak i bardzo zaawansowane algorytmy, zawierające wiele dodatkowych mechanizmów usprawniających ich działanie.

Kolejnym etapem było przygotowanie modelu programowego dla każdego algorytmu. Wszystkie implementacje przygotowano z wykorzystaniem biblioteki \textit{OpenCV} i uruchomiono na tym samym komputerze klasy PC. Do przetestowania zaimplementowanych rozwiązań wykorzystano bardzo często stosowaną metodologię \cite{} oraz zbiór sekwencji testowych \textit{Change Detection} \cite{}. Otrzymane wyniki były momentami zaskakujące, okazało się między innymi, że w wielu przypadkach bardziej zaawansowane algorytmy dają wyraźnie gorsze rezultaty niż o wiele mniej skomplikowane rozwiązania. Niestety, nie każde z przedstawionych podejść okazało się być idealnym rozwiązaniem. Zaproponowany mechanizm eliminacji drgań kamery nie sprawdził się i nie zapewnił zadowalających wyników w testach z wykorzystaniem wspomnianych zbirów testowych. Efekty tego rozwiązania były jednak widoczne podczas działania algorytmu w rzeczywistym środowisku. 

Następny i jednocześnie najbardziej zaawansowany etap pracy to przygotowanie implementacji sprzętowej każdego z algorytmów. Przygotowane modele programowe nie dają możliwości przetwarzania obrazu w czasie rzeczywistym. Maksymalna prędkość analizy obrazu zależy od stopnia złożoności algorytmu i waha się w przedziale od jednej do kilkunastu klatek na sekundę. W związku z powyższym, wszystkie algorytmy zaimplementowano na wspólnej platformie sprzętowej i porównano pod kątem optymalizacji i zużycia zasobów. Podczas tworzenia implementacji posiłkowano się rozwiązaniami zaimplementowanymi już w Laboratorium Biocybernetyki AGH. Większość metod udało się uruchomić we wszystkich wymaganych rozdzielczościach, od \textit{576p@50fps} i \textit{1080p@50fps}. W pojedynczych przypadkach ze względu stopień rozbudowania algorytmu, zaprzestano na podstawowej rozdzielczości.

Podsumowując, przygotowana biblioteka zawiera wiele ciekawych i efektywnych algorytmów, a~uzyskane wyniki testów są zbliżone do rezultatów innych popularnych metod z rankingu \textit{Change Detection} \cite{}. Przeniesienie implementacji na układ \textit{FPGA} w niektórych przypadkach było kłopotliwe, między innymi ze względu na występujące ograniczenia sprzętowe. Ostatecznie jednak wszystkie metody udało się zaimplementować i pomyślnie uruchomić. Pomimo osiągniętego sukcesu nadal istniej wiele możliwości rozwoju stworzonej biblioteki. Jak zostało wspomniane w rozdziale \ref{cha:kierunki_rozwoju}, można rozszerzyć algorytmy o dodatkową funkcjonalność jak na przykład, detekcja cieni i eliminacja dynamicznego tła. Kolejnym kierunkiem rozwoju jest optymalizacja i wzrost wydajności oraz dostosowanie wszystkich metod do przetwarzania obrazu w rozdzielczości \textit{Full HD} lub wyższej. Zakres nowych możliwości i~dalszych prac jest bardzo szeroki, jednak biorąc pod uwagę przyjęte na początku cele, uzyskane efekty można z~pewnością uznać za zadowalające. 