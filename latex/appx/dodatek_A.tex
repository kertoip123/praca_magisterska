
\chapter{Spis zawartości płyty DVD}
\label{appx:plyta}

\begin{enumerate}
\item{Praca zapisana w formacie \textit{\small{PDF}}.}
\item{Praca zapisana w systemie \LaTeX.}
\item{Wszystkie rysunki użyte w pracy, zapisane w formacie \textit{\small{PNG}} lub \textit{\small{PDF}}.}
\item{
		Sekwencje testowe pobrane ze strony \underline{\texttt{\small{http://www.changedetection.net}}}:
		\begin{itemize}
			\setlength\itemsep{0em}
			\item{\textit{Baseline}}
			\item{\textit{Dynamic Background}}
			\item{\textit{Camera Jitter}}
			\item{\textit{Intermittent Object Motion}}
			\item{\textit{Shadows}}
			\item{\textit{Thermal}}
		\end{itemize}
}
\item{
		Projekt w środowisku \textit{Microsoft Visual Studio 2015} zawierający implementację:
		\begin{itemize}
			\setlength\itemsep{0em}
			\item{prostej \textit{metody naiwnej}, \textit{odejmowania ramek} oraz \textit{średniej kroczącej}}
			\item{algorytmu \textit{GMM}}
			\item{algorytmu \textit{ViBE} wraz z dodatkowym modułem niwelującym ruch kamery}
			\item{algorytmu \textit{PBAS} wraz z dodatkowych mechanizmem wykrywania obiektów statycznych}
			\item{środowiska do przeprowadzania testów i wyznaczania współczynników jakości}
		\end{itemize}				
}
\item{Projekty w środowisku \textit{Vivado 2017.1} zawierający implementację sprzętową następujących algorytmów (w nawiasach podano nazwę projektu):
		\begin{itemize}
			\setlength\itemsep{0em}
			\item{\textit{metoda naiwnej}, \textit{odejmowania ramek} oraz \textit{średnia krocząca} (\textit{hdmi\_vc707})}
			\item{\textit{GMM} (\textit{gmm\_vc707})}
			\item{\textit{ViBE} (\textit{vibe\_vc707})}
			\item{\textit{ViBE} w wysokiej rozdzielczości (\textit{vibe\_high\_res\_vc707})}
			\item{rozszerzona wersja \textit{ViBE} (\textit{vibe\_plus\_vc707})}
			\item{\textit{PBAS} (\textit{pbas\_vc707})}
			\item{\textit{PBAS}  w wysokiej rozdzielczości (\textit{pbas\_high\_res\_vc707})}
			\item{rozszerzona wersja \textit{PBAS} (\textit{pbas\_plus\_vc707})}
		\end{itemize}	
}
\end{enumerate}